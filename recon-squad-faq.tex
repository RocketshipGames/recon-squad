%%----------------------------------------------------------------------
%%--------------------------------------------------------------
\missiontitle{Frequently Asked Questions}
\hrule
\smallskip
\hrule

\bigskip
%\begin{story}{1in}{Priceless Relics}
%
%\end{story}

\begin{multicols}{2}


% The following are common or expected questions about Recon Squad's
% special play style and rules.  Note that these themselves are not
% normative rules, they're just providing more specific notes and
% examples on how the Recon Squad rules play out.

% Many of these are due to the Army Of One rule dividing standard
% multi-model units into singe model units in Recon Squad.  This is
% problematic when a single model, like an upgrade character, or
% individual piece of wargear provides special rules to the whole squad.
% Those rules would no longer apply to any models except the special one
% or wargear bearer, as they would would be in a separate unit from
% their squadmates.  However, in those cases, the Side Effects rule for
% Recon Squad enables those models from the squad within 3" of the
% source to use the effect.  It also applies those rules before
% deployment, e.g., to Infiltrate.


\missionheading{Army Selection}%

\smallskip\noindent\emph{Do normal build requirements apply other than
  the restrictions here? For example, do I need to select a Necron
  Overloard to take a Royal Court?  Or a Space Marine Captain to take
  a Command Squad?}

Yes, all of these rules still apply.


\smallskip\noindent\emph{Are units that don't occupy a FOC slot, like
  Space Marine Servitors, Necron Royal Courts, and dedicated
  transports, allowed?}

Yes, following the usual restrictions and rules.


\bigskip\missionheading{Traits}%

\smallskip\noindent\emph{Does choosing Scout as a specialist trait
  permit the model to be placed into Reserve and then Outflank?}

- No.  Only mission specific rules and the Tactical Genius strategem
  permit placing units in reserve, as per the No Holding Back rule.

  \smallskip\noindent\emph{Do Sniper and Tank Hunter only apply to one
    weapon when chosen as specialist traits, like many of the other
    traits?}

- No, they are intentionally not marked as such.  All the model's
  shooting and close combat attacks benefit.

\smallskip\noindent\emph{When exactly are stratagems declared?}

- After the winner of the roll-off to determine deployment order has
  decided to go first or second, but before either player begins any
  deployment.  I.e., you will know whether or not Just As Planned is
  applicable for you.

\bigskip\missionheading{Game Rules}%

\smallskip\noindent\emph{Are conjurations permitted?}

- No, units may not be summoned.  Help's Not Coming rule of Recon
  Squad prohibits all conjurations.

\smallskip\noindent\emph{Can witchfire shots be divided as per Recon Squad's Unload rule?}

- Yes.

\bigskip\missionheading{Necrons}%

\smallskip\noindent\emph{How do Necron Reanimation Protocols and
  Resurrection Orbs work here?}

- Following the Codex Specific Adaptations for Necrons, casualties are
  marked directly rather than putting a counter on their unit since
  the latter doesn't exist in Recon Squad due to the Army Of One rule.
  Reanimation Protocol rolls may be made for those marked casualties
  within 3" of a friendly Necron model not also marked as a casualty.
  On success the model stands back up with 1 wound remaining and the
  counter is removed.  On failure both model and counter are removed.

  Applying the Recon Squad Side Effects rule, if the marked model was
  part of a unit selection in the original army list with a
  Resurrection Orb, its Reanimation Roll passes on a 4+ instead of a
  5+ if the Resurrection Orb is within 3" of the marked casualty.

  \smallskip\noindent\emph{Is line of sight or movement blocked by
    Necron models killed but marked with counters to potentially
    reanimate instead of removing?}

- No.  If enemy models move within 1" of the marked model simply move
  it and its counter the shortest distance to place it 1" away from
  all enemy models and not in impassible terrain.


\bigskip\missionheading{Astra Militarum}%

\smallskip\noindent\emph{How does Commissars' Summary Execution
  apply?}

- Due to the Side Effects rule, anytime a model from the Commissar's
  original army list selection fails a morale test while within 3" of
  the Commissar, that model is removed as a casualty.  That's pretty
  harsh, so Guardsmen worried they might look weak in front of the
  Commissar should stay 7" away!  Note that due to the We All Die
  Alone rule, Independent Character Commissars simply don't apply
  Summary Execution---they cannot be part of a unit.

\smallskip\noindent\emph{How does Sgt Harker work, given the Army Of One rule?}

- The Side Effects rule grants models from his squad the Stealth and
  Move Through Cover USRs whenever he is within 3".  In addition it
  enables all members of his squad to Infiltrate during deployment.

\smallskip\noindent\emph{How do Medi-Packs work?}

- Any models from the medic's original army list selection have Feel
  No Pain while within 3" of him, due to the Side Effects rule.

\bigskip\missionheading{Orks}%

\smallskip\noindent\emph{How do Boss Poles work?}

- Due to the Side Effects rule, whenever a model from an army list
  selection including a Bosspole fails a morale test while within 3"
  of the Nob with the Bosspolle, that model may take a wound (and the
  save against it) in order to reroll the morale test.

\smallskip\noindent\emph{How does Snikrot work?}

- All the models from his original army list selection would have the
  Move Through Cover rule while within 3" of Snikrot.  Unless the
  mission notes otherwise, they would not be able to deploy in reserve
  due to the No Holding Back rule.  However, if the missions permits
  reserves, note otherwise, they would all be able to enter from any
  table edge as the Side Effects rule confers his Ambush rule to the
  other units in his army list selection before deployment.
  Similarly, when deploying on the table, they may all Infiltrate.


\bigskip\missionheading{Daemons}%

\smallskip\noindent\emph{How do Pink Horrors of Tzeentch work?}

- Due to the Army Of One rule, Brotherhood of Psykers/Sorcerors would
  not apply directly.  However, the Cast A Spell On You rule adapts
  Brotherhood such that on each turn, one Pink Horror per army list
  selection would be able to make the Horrors' psychic shooting
  attack.  This is resolved as usual.


\bigskip\missionheading{Space Marines}%

\smallskip\noindent\emph{How does Sgt Telion work?}

- Like a badass.  Any scout from his squad within 3" of him receives
  the Stealth USR.  Further, if he does not shoot or run in a shooting
  phase, one scout within 3" of him may use his ballistic skill, per
  his Voice of Experience rule and as adapted by Side Effects.

\end{multicols}
